\documentclass{article}

\usepackage[utf8]{inputenc}
\usepackage{enumitem}
\usepackage{amsmath}
\usepackage{amsthm}
\usepackage{amssymb}
\usepackage{graphicx}
\usepackage{clrscode3e}

\newcommand{\mname}[1]{\mbox{\sf #1}}
\newcommand{\pnote}[1]{{\langle \text{#1} \rangle}}

\title{Assignment 1}
\author{Nguyen Gia Hien Tu - tun1}
\date{\today}

\begin{document}

\maketitle

\textbf{Analysis} \\

The first paragraph of the description is ignored when creating the ER diagram
since it is just an introduction.

The sentence ``First, the system will store information about several cinemas''
indicates that there is an entity Cinema for the diagram. ``Each cinema has
a unique name and anaddress'' shows that the entity Cinema has attributes name
and address. Although each cinema has a unique name, cinema names have changed
in the past. Thus, we cannot let the attribute name be the primary key. We cannot
let the attribute address be the primary key either since the location of a
cinema may change (e.g., moving the cinema to another place or expanding the
cinema). Thus, I believe it is better to add another attribute called cid, stands
for cinema id, and let this attribute be the primary key. Therefore, the Cinema
entity has three cid, name and address where cid is the primary key.

``Per cinema, the system will also maintain information per room'' illustrates
that the entity Room is a owned by the entity Cinema. In other words, Room is a
weak entity of Cinema since a room can only be uniquely defined when there is a
cinema corresponding with it. Since each room has a different room number, the
attribute room\_num is added to be the partial key for the entity Room. So, the
primary key for the entity Room is the pair (cid, room\_num). Based on the 
description, the entity also has attributes screen\_type, screen\_size,
projector\_type, sound\_system and accessibility. The examples of screen type,
projector type, sound system are ignored since we focus on the attributes rather
than the examples of each attribute when creating the diagram.

``This information is not only available for cinema visitors, but will also be
communicated to private partiesthat are looking to hire a room (e.g., for a
corporate event)'' is ignored since the visibility is not restricted to anyone
and 

 

\end{document}