\documentclass[12pt]{article}

\usepackage[utf8]{inputenc}
\usepackage{enumitem}
\usepackage{amsmath}
\usepackage{amsthm}
\usepackage{amssymb}
\usepackage{graphicx}
\usepackage{tikz}
\usepackage[normalem]{ulem}

\newcommand{\mname}[1]{\mbox{\sf #1}}
\newcommand{\pnote}[1]{{\langle \text{#1} \rangle}}


\title{Assignment 5: Dependency Theory}
\author{Hien Tu - tun1}
\date{\today}

\begin{document}

\maketitle

\textbf{D1 Reasoning with dependencies}

\begin{enumerate}
  % Question 1
  \item
    \begin{itemize}
      \item Assume $AB \longrightarrow C, A \longrightarrow D$ and $CD
            \longrightarrow EF$
      \item Apply Augmentation on $AB \longrightarrow C$ with $A$ to derive $AB
            \longrightarrow CA$
      \item Apply Augmentation on $A \longrightarrow D$ with $C$ to dervie $CA
            \longrightarrow CD$
      \item Apply Transitivity on $AB \longrightarrow CA$ and $CA
            \longrightarrow CD$ to derive $AB \longrightarrow CD$
      \item Apply Transitivity on $AB \longrightarrow CD$ and $CD
            \longrightarrow EF$ to derive $AB \longrightarrow EF$
      \item Apply Reflection on $F \subseteq EF$ to derive $EF \longrightarrow
            F$
      \item Apply Transitivity on $AB \longrightarrow EF$ and $EF
            \longrightarrow F$ to derive $AB \longrightarrow F$
      \item Hence, $AB \longrightarrow F$
    \end{itemize}
    \ \\

  % Question 2
  \item
    \begin{itemize}
      \item Assume we have rows $r_1, r_2 \in I$ of instance $I$ such that
            $r_1[XW] = r_2[XW]$
      \item By $r_1[XW] = r_2[XW]$, we have $r_1[X] = r_2[X]$ and $r_1[W] =
            r_2[W]$
      \item Using $X \longrightarrow Y$ and $r_1[X] = r_2[X]$, we conclude that
            $r_1[Y] = r_2[Y]$
      \item By $r_1[Y] = r_2[Y]$ and $r_1[W] = r_2[W]$, we have $r_1[YW] =
            r_2[YW]$
      \item Using $YW \longrightarrow Z$ and $r_1[YW] = r_2[YW]$, we conclude
            that $r_1[Z] = r_2[Z]$
      \item Hence, $r_1[Z] = r_2[Z]$ holds.
    \end{itemize}
  \ \\

  % Question 3
  \item
    \begin{itemize}
      \item Assume we have $r_1[X]$ for every instance $I_1$ of $R$ and every
            row $r_1 \in I_1$
      \item Using $R[X] \subseteq S[Y]$ and $r_1[X]$, there exists a row in
            instance $I_2$ of $S$ with $r_1[X] = r_2[X]$
      \item Using $S[Y] \subseteq T[Z]$ and $r_2[Y]$, there exists a row in
            instance $I_3$ of $T$ with $r_2[Y] = r_3[Z]$
      \item Thus, for every instance $I_1$ of $R$ and every row $r_1 \in R$,
            there exists a row in instance $I_3$ of $T$ such that $r_1[X] =
            r_3[Z]$
      \item Hence, $R[X] \subseteq T[Z]$
    \end{itemize}
  \ \\

  % Question 4
  \item
    \begin{itemize}
      \item Assume $X \twoheadrightarrow Y$ and $XY \longrightarrow Z$
      \item Apply Complementation on $X \twoheadrightarrow Y$ to derive $X
            \twoheadrightarrow Z$ (with $Z$ all attributes of $\textbf{R}$ not
            in $X$ and $Y$)
      \item Apply Reflexivity on $Z \setminus (X \cup Y) \subseteq Z$ to derive
            $Z \longrightarrow Z \setminus (X \cup Y)$
      \item Apply Transitivity on $XY \longrightarrow Z$ and $Z \longrightarrow
            Z \setminus (X \cup Y)$ to derive $XY \longrightarrow Z \setminus (X
            \cup Y)$
      \item Since $Z$ is all attributes of $\textbf{R}$ not in $X$ and $Y$, $Z
            \cap XY = \emptyset$
      \item Apply Coalensence on $X \twoheadrightarrow Z$, $XY \longrightarrow Z
            \setminus (X \cup Y)$, $Z \cap XY = \emptyset$ and $Z \setminus (X
            \cup Y) \subseteq Z$, we conclude that $X \longrightarrow Z
            \setminus (X \cup Y)$
      \item Hence, $X \longrightarrow Z \setminus (X \cup Y)$
    \end{itemize}
  \ \\

  % Question 5
  \item Consider the following table of the schema
        \begin{center}
          \textbf{person}(\underline{name}, \underline{number}, birthdate, age)
        \end{center}
        \begin{center}
          \begin{tabular}{ | c | c | c | c | }
            \hline
            \textbf{name} & \textbf{number} & \textbf{birthdate} & \textbf{age} \\
            \hline
            Alice         & 1               & 2001-01-01        & 20 \\
            Alice         & 2               & 2005-09-05        & 16 \\
            \hline
          \end{tabular}
        \end{center}
        Since \underline{name}, \underline{number} are the primary keys, they       
        determine all attributes. Thus, ``name, number $\longrightarrow$
        birthdate''. \\
        We know from the lecture that birthdate determines age since people who
        have the same birthdate would have the same age, and so ``birthdate
        $\longrightarrow$ age''. \\
        Since \{birthdate\} $\subseteq$ \{name, birthdate\}, by applying
        Reflexivity, we get ``name, birthdate $\longrightarrow$ birthdate''. \\
        Applying Transitivity on ``name, birthdate $\longrightarrow$ birthdate''
        and ``birthdate $\longrightarrow$ age'', we derive ``name, birthdate
        $\longrightarrow$ age'' \\
        Hence, we have ``name, number $\longrightarrow$ birthdate'' and ``name,
        birthdate $\longrightarrow$ age''. \\
        However, from the table, we can see that ``name $\longrightarrow$ age''
        does not hold. \\
        Let $X =$ name, $W =$ number, $Y =$ birthdate, $Z =$ age, we have shown
        that the inference rule from the question is not sound.
  \ \\

  % Question 6
  \item TODO: Do Question 6
  \ \\

  % Question 7
  \item The attribute closure of set of attributes $C$:
        \begin{itemize}
          \item Initially, $closure = \{C\}$
          \item From $C \longrightarrow A$, since $C \subseteq closure$ and $A
                \not \subseteq closure$, $closure = \{C, A\}$
          \item From $AC \longrightarrow E$, since $AC \subseteq closure$ and
                $E \not \subseteq closure$, $closure = \{C, A, E\}$
          \item From $E \longrightarrow B$, since $E \subseteq closure$ and $B
                \not \subseteq closure$, $closure = \{C, A, E ,B\}$
          \item From $AB \longrightarrow D$, since $AB \subseteq closure$ and $D
                \not \subseteq closure$, $closure = \{C, A, E, B, D\}$
          \item From $BC \longrightarrow D$, since $BC \subseteq closure$ and $D
                \subseteq closure$, we don't need to add $D$ into $closure$ more
                (as $D$ is already in $closure$)
          \item From $D \longrightarrow A$, since $D \subseteq closure$ and $A
                \subseteq closure$, we don't need to add $A$ into $closure$ more
                (as $A$ is already in $closure$)
          \item Therefore, $C^+ = \{A, B, C, D, E\}$
        \end{itemize}

        The attribute closure of set of attributes $(EA)$:
        \begin{itemize}
          \item Initially, $closure = \{E, A\}$
          \item From $E \longrightarrow B$, since $E \subseteq closure$ and $B
                \not \subseteq closure$, $closure = \{E, A, B\}$
          \item From $AB \longrightarrow D$, since $AB \subseteq closure$ and $D
                \not \subseteq closure$, $closure = \{E, A, B, D\}$
          \item From $AC \longrightarrow E$, since $AC \not \subseteq closure$,
                we don't need to add $E$ into $closure$
          \item From $BC \longrightarrow D$, since $BC \not \subseteq closure$,
                we don't need to add $D$ into $closure$
          \item From $C \longrightarrow A$, since $C \not \subseteq closure$,
                we don't need to add $A$ into $closure$
          \item From $D \longrightarrow A$, since $D \subseteq closure$ and $A
                \subseteq closure$, we don't need to add $A$ into $closure$ more
                (as $A$ is already in $closure$)
          \item Therefore, $(EA)^+ = \{A, B, D, E\}$
        \end{itemize}
  \ \\
    
  % Question 8
  \item Compute $X^+$ for every $X \subseteq \{A, B, C, D, E\}$
        \begin{itemize}
          \item $A^+ = \{A\}$ since there is no dependency that would
                satisfy the closure algorithm \\
                So we have $A \longrightarrow A$
          \item $B^+ = \{B\}$ since there is no dependency that would
                satisfy the closure algorithm \\
                So we have $B \longrightarrow B$
          \item $C^+ = \{A, B, C, D, E\}$ from question 7 \\
                So we have $C \longrightarrow X$ for all $X \subseteq \{A, B, C,
                D, E\}$
          \item $D^+ = \{A, D\}$ since there is only $D \longrightarrow A$ that
                would satisfy the closure algorithm \\
                So we have $D \longrightarrow X$ for all $X \subseteq \{A, D\}$
          \item $E^+ = \{B, E\}$ since there is only $E \longrightarrow B$ that
                would satisfy the closure algorithm \\
                So we have $E \longrightarrow X$ for all $X \subseteq \{B, E\}$
          \item $(AB)^+ = \{A, B, D\}$ since there is only $AB \longrightarrow
                D$ that would satisfy the closure algorithm \\
                So we have $AB \longrightarrow X$ for all $X \subseteq \{A, B,
                D\}$
          \item $(AC)^+ = \{A, B, C, D, E\}$ since $AC$ includes $C$ and we can
                reach others from only $C$ \\
                So we have $AC \longrightarrow X$ for all $X \subseteq \{A, B,
                C, D, E\}$
          \item $(AD)^+ = \{A, D\}$ since there is no dependency that would
                satisfy the closure algorithm \\
                So we have $AD \longrightarrow X$ for all $X \subseteq \{A, D\}$
          \item $(AE)^+ = \{A, B, E\}$ since there is only $E \longrightarrow B$
                that would satisfy the closure algorithm \\
                So we have $AE \longrightarrow X$ for all $X \subseteq \{A, B,
                E\}$
          \item $(BC)^+ = \{A, B, C, D, E\}$ since $BC$ includes $C$ and we can
                reach others from only $C$ \\
                So we have $BC \longrightarrow X$ for all $X \subseteq \{A, B,
                C, D, E\}$
          \item $(BD)^+ = \{A, B, D\}$ since there is only $D \longrightarrow A$
                that would satisfy the closure algorithm \\
                So we have $BD \longrightarrow X$ for all $X \subseteq \{A, B, D
                \}$
          \item $(BE)^+ = \{B, E\}$ since there is no dependency that would
                satisfy the closure algorithm \\
                So we have $BE \longrightarrow X$ for all $X \subseteq \{B, E\}$
          \item $(CD)^+ = \{A, B, C, D, E\}$ since $CD$ includes $C$ and we can
                reach others from only $C$ \\
                So we have $CD \longrightarrow X$ for all $X \subseteq \{A, B,
                C, D, E\}$
          \item $(CE)^+ = \{A, B, C, D, E\}$ since $CE$ includes $C$ and we can
                reach others from only $C$ \\
                So we have $CE \longrightarrow X$ for all $X \subseteq \{A, B,
                C, D, E\}$
          \item $(DE)^+ = \{A, B, D, E\}$ there are $D \longrightarrow A$, $E
                \longrightarrow B$ satisfy the closure algorithm \\
                So we have $DE \longrightarrow X$ for all $X \subseteq \{A, B,
                D, E\}$
          \item $(ABC)^+ = \{A, B, C, D, E\}$ since $ABC$ includes $C$ \\
                So we have $ABC \longrightarrow X$ for all $X \subseteq \{A, B,
                C, D, E\}$
          \item $(ABD)^+ = \{A, B, D\}$ since there is no dependency that would
                satisfy the closure algorithm \\
                So we have $ABD \longrightarrow X$ for all $X \subseteq \{A, B,
                D\}$
          \item $(ABE)^+ = \{A, B, D, E\}$ since there is only $AB
                \longrightarrow D$ that would satisfy the closure algorithm \\
                So we have $ABE \longrightarrow X$ for all $X \subseteq \{A, B,
                D, E\}$
          \item $(ACD)^+ = \{A, B, C, D, E\}$ since $ACD$ includes $C$ \\
                So we have $ACD \longrightarrow X$ for all $X \subseteq \{A, B,
                C, D, E\}$
          \item $(ACE)^+ = \{A, B, C, D, E\}$ since $ACE$ includes $C$ \\
                So we have $ACE \longrightarrow X$ for all $X \subseteq \{A, B,
                C, D, E\}$
          \item $(ADE)^+ = \{A, B, D, E\}$ since there is only $E
                \longrightarrow B$ that would satisfy the closure algorithm \\
                So we have $ADE \longrightarrow X$ for all $X \subseteq \{A, B,
                D, E\}$
          \item $(BCD)^+ = \{A, B, C, D, E\}$ since $BCD$ includes $C$ \\
                So we have $BCD \longrightarrow X$ for all $X \subseteq \{A, B,
                C, D, E\}$
          \item $(BCE)^+ = \{A, B, C, D, E\}$ since $BCE$ includes $C$ \\
                So we have $BCE \longrightarrow X$ for all $X \subseteq \{A, B,
                C, D, E\}$
          \item $(BDE)^+ = \{A, B, D, E\}$ since there is only $D 
                \longrightarrow A$ that would satisfy the closure algorithm \\
                So we have $BDE \longrightarrow X$ for all $X \subseteq \{A, B,
                D, E\}$
          \item $(CDE)^+ = \{A, B, C, D, E\}$ since $CDE$ includes $C$ \\
                So we have $CDE \longrightarrow X$ for all $X \subseteq \{A, B,
                C, D, E\}$
          \item $(ABCD)^+ = \{A, B, C, D, E\}$ since $ABCD$ includes $C$ \\
                So we have $ABCD \longrightarrow X$ for all $X \subseteq \{A, B,
                C, D, E\}$
          \item $(ABCE)^+ = \{A, B, C, D, E\}$ since $ABCE$ includes $C$ \\
                So we have $ABCE \longrightarrow X$ for all $X \subseteq \{A, B,
                C, D, E\}$
          \item $(ABDE)^+ = \{A, B, D, E\}$ since there is no dependency that
                would satisfy the closure algorithm \\
                So we have $ABDE \longrightarrow X$ for all $X \subseteq \{A, B,
                C, D, E\}$
          \item $(ACDE)^+ = \{A, B, C, D, E\}$ since $ACDE$ includes $C$ \\
                So we have $ACDE \longrightarrow X$ for all $X \subseteq \{A, B,
                C, D, E\}$
          \item $(BCDE)^+ = \{A, B, C, D, E\}$ since $BCDE$ includes $C$ \\
                So we have $BCDE \longrightarrow X$ for all $X \subseteq \{A, B,
                C, D, E\}$
          \item $(ABCDE)^+ = \{A, B, C, D, E\}$ since $ABCDE$ includes $C$ \\
                So we have $ABCDE \longrightarrow X$ for all $X \subseteq \{A,
                B, C, D, E\}$

          \item Any combination that contains $C$ is the superkey
          \item $C$ is the (candidate) key
        \end{itemize}
  \ \\

  % Question 9
  \item
    \begin{itemize}
      \item Starting with $\{AB \longrightarrow D, AC \longrightarrow E, BC
            \longrightarrow D, C \longrightarrow A, D \longrightarrow A, E
            \longrightarrow B\}$
      \item From $C \longrightarrow A$, we can use Augmentation with $C$ to
            derive $C \longrightarrow AC$. We also have $AC \longrightarrow E$.
            Then, by Transitivity, we conclude $C \longrightarrow E$. Thus, we
            can add $C \longrightarrow E$ to our set. \\
            $\{AB \longrightarrow D, AC \longrightarrow E, BC \longrightarrow
            D, C \longrightarrow A, D \longrightarrow A, E \longrightarrow B, C
            \longrightarrow E\}$
      \item From $C \longrightarrow E$, we can use Augmentation with $A$ to
            derive $AC \longrightarrow AE$. Then, we can use Decomposition to
            get $AC \longrightarrow A$ and $AC \longrightarrow E$. $AC
            \longrightarrow A$ is trivial since $A \subseteq AC$. This means
            that, from $C \longrightarrow E$, we can get $AC \longrightarrow E$.
            So, we can get rid of $AC \longrightarrow E$. \\
            $\{AB \longrightarrow D, BC \longrightarrow D, C \longrightarrow A,
            D \longrightarrow A, E \longrightarrow B, C \longrightarrow E\}$
      \item From $C \longrightarrow E$ and $E \longrightarrow B$, by
            Transitivity, we get $C \longrightarrow B$. Then, we get apply
            Augmentation on $C \longrightarrow B$ with $C$ to derive $C
            \longrightarrow BC$. We also have $BC \longrightarrow D$. So, by
            Transitivity, we derive $C \longrightarrow D$. Thus, we can add $C
            \longrightarrow D$ to our set. \\
            $\{AB \longrightarrow D, BC \longrightarrow D, C \longrightarrow A,
            D \longrightarrow A, E \longrightarrow B, C \longrightarrow E, C
            \longrightarrow D\}$
      \item From $C \longrightarrow D$, we can apply Augmentation with $B$ to
            get $BC \longrightarrow BD$. Then, we can use Decomposition to get
            $BC \longrightarrow B$ and $BC \longrightarrow D$. $BC
            \longrightarrow B$ is trivial since $B \subseteq BC$. This means
            that from $C \longrightarrow D$, we can get $BC \longrightarrow D$.
            So, we can get rid of $BC \longrightarrow D$. \\
            $\{AB \longrightarrow D, C \longrightarrow A, D \longrightarrow A, E
            \longrightarrow B, C \longrightarrow E, C \longrightarrow D\}$
      \item By Transitivity on $C \longrightarrow D$ and $D \longrightarrow A$,
            we can derive $C \longrightarrow A$. Thus, we can get rid of $C
            \longrightarrow A$. \\
            $\{AB \longrightarrow D, D \longrightarrow A, E \longrightarrow B, C
            \longrightarrow E, C \longrightarrow D\}$
    \end{itemize}
  
  

\end{enumerate}

\end{document}