\documentclass{article}

\usepackage[utf8]{inputenc}
\usepackage{enumitem}
\usepackage{amsmath}
\usepackage{amsthm}
\usepackage{amssymb}
\usepackage{graphicx}
\usepackage[normalem]{ulem}

\newcommand{\key}[1]{\underline{\smash{#1}}}
\newcommand{\pkey}[1]{\dashuline{\smash{#1}}}
\newcommand{\mname}[1]{\mbox{\sf #1}}
\newcommand{\pnote}[1]{{\langle \text{#1} \rangle}}

\title{Assignment 3}
\author{Hien Tu - tun1}
\date{\today}

\begin{document}

\maketitle

\textbf{Part One: The social media features}

From the ER-diagram, there is the entity Subscriber with the primary keys
\key{username} and \key{number}. Thus, we will have a table named Subscriber and
we will make (\key{username}, \key{number}) to be the primary key pair. There
are also attributes email, hash and salt in the entity Subscriber. We will use
large strings CLOB for \key{username}. For \key{number}, INT is used for the
domain of attribute. Since both \key{username} and \key{number} are used as
primary keys, two different subscribers can either have the same username or the
same number but not both (exclusive or). We would use VARCHAR(320) to store
email. Since hash and salt values require 64-byte each, we would use BINARY(64)
for each of them. Since all the above attributes are required to create a
subscriber, all of them need to be not null (except for \key{number} since it is
already automatically incremented). The relational schema for Subscriber table
is
\begin{align*}
    \text{\textbf{Subscriber}(} & \text{\key{username}: CLOB,} \\
                                & \text{\key{number}: INT,} \\
                                & \text{email: VARCHAR(320),} \\
                                & \text{hash: BINARY(64),} \\
                                & \text{salt: BINARY(64)).}
\end{align*}

The ER-diagram has a many-to-many relationship Friend\_Of between Subscriber and
Subscriber. Thus, we will have a table name Friend\_Of. The primary keys of the
Subscriber will be the primary keys of the Friend\_Of. Since there are two
subscribers, we will have a 4-tuple primary key (\key{funame}, \key{funum},
\key{tuname}, \key{tunum}). \key{funame} is short for from username, which is
the username of the subscriber who is following and \key{funum} is short for
from user number, which is the number of the subscriber who is following.
Similarly, \key{tuname} and \key{tunum} is the username and the user number of
the subscriber who is being followed, respectively. Thus, \key{funame} and
\key{tuname} would reference \key{username} (column) of the Subscriber table
while \key{funum} and \key{tunum} would reference \key{number} (column) of the
Subscriber table. Since the domain of the attribute \key{username} in the
Subscriber table is CLOB, we need to use CLOB to be the domain of the attributes
(keys) \key{funame} and \key{tuname}. Similarly, we need to use INT for
\key{funum} and \key{tunum}. All of the attributes would be not null as well. We
also need to check that a subscriber is not following themselves, that is, we
need to check either \key{funame} $<>$ \key{tuname} or \key{funum} $<>$
\key{tunum} or both are different (inclusive or). The relational schema for the
Friend\_Of table is
\begin{align*}
    \text{\textbf{Friend\_Of}(} & \text{\key{funame}: CLOB,} \\
                                & \text{\key{funum}: INT,} \\
                                & \text{\key{tuname}: CLOB,} \\
                                & \text{\key{tunum}: INT).} 
\end{align*}

We also have the entity Reaction, which will be the table Reaction in the
relational schema. Since Reaction has a one-to-many relationship with Subscriber,
we can add the primary keys of the table Subscriber to be the attributes of the
table Reaction to represent this relationship. We name these attributes to be
uname and unum correspond to the \key{username} and \key{number} in the table
Subscriber with the domain of CLOB and INT, respectively.



\end{document}