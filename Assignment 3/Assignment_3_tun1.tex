\documentclass{article}

\usepackage[utf8]{inputenc}
\usepackage{enumitem}
\usepackage{amsmath}
\usepackage{amsthm}
\usepackage{amssymb}
\usepackage{graphicx}
\usepackage[normalem]{ulem}

\newcommand{\key}[1]{\underline{\smash{#1}}}
\newcommand{\pkey}[1]{\dashuline{\smash{#1}}}
\newcommand{\mname}[1]{\mbox{\sf #1}}
\newcommand{\pnote}[1]{{\langle \text{#1} \rangle}}

\title{Assignment 3}
\author{Hien Tu - tun1}
\date{\today}

\begin{document}

\maketitle

\textbf{Part One: The social media features}

From the ER-diagram, there is the entity Subscriber with the primary keys
\key{username} and \key{number}. Thus, we will have a table named Subscriber and
we will make (\key{username}, \key{number}) to be the primary key pair. There
are also attributes email, hash and salt in the entity Subscriber. We will use
large strings CLOB for \key{username}. For \key{number}, INT is used for the
domain of attribute. Since both \key{username} and \key{number} are used as
primary keys, two different subscribers can either have the same username or the
same number but not both (exclusive or). We would use VARCHAR(320) to store
email. Since hash and salt values require 64-byte each, we would use BINARY(64)
for each of them. Since all the above attributes are required to create a
subscriber, all of them need to be not null (except for \key{number} since it is
already automatically incremented). The relational schema for Subscriber table
is
\begin{align*}
    \text{\textbf{Subscriber}(} & \text{\key{username}: CLOB,} \\
                                & \text{\key{number}: INT,} \\
                                & \text{email: VARCHAR(320),} \\
                                & \text{hash: BINARY(64),} \\
                                & \text{salt: BINARY(64)).}
\end{align*}

The ER-diagram has a many-to-many relationship Friend\_Of between Subscriber and
Subscriber. Thus, we will have a table name Friend\_Of. The primary keys of the
Subscriber will be the primary keys of the Friend\_Of. Since there are two
subscribers, we will have a 4-tuple primary key (\key{funame}, \key{funum},
\key{tuname}, \key{tunum}). \key{funame} is short for from username, which is
the username of the subscriber who is following and \key{funum} is short for
from user number, which is the number of the subscriber who is following.
Similarly, \key{tuname} and \key{tunum} is the username and the user number of
the subscriber who is being followed, respectively. Thus, \key{funame} and
\key{tuname} would reference \key{username} (column) of the Subscriber table
while \key{funum} and \key{tunum} would reference \key{number} (column) of the
Subscriber table. Since the domain of the attribute \key{username} in the
Subscriber table is CLOB, we need to use CLOB to be the domain of the attributes
(keys) \key{funame} and \key{tuname}. Similarly, we need to use INT for
\key{funum} and \key{tunum}. All of the attributes would be not null as well. We
also need to check that a subscriber is not following themselves, that is, we
need to check either \key{funame} $<>$ \key{tuname} or \key{funum} $<>$
\key{tunum} or both are different (inclusive or). The relational schema for the
Friend\_Of table is
\begin{align*}
    \text{\textbf{Friend\_Of}(} & \text{\key{funame}: CLOB,} \\
                                & \text{\key{funum}: INT,} \\
                                & \text{\key{tuname}: CLOB,} \\
                                & \text{\key{tunum}: INT).} 
\end{align*}

We also have the entity Reaction, which will be the table Reaction in the
relational schema. Notice that Reaction, ThreadR, and ReviewR has an ISA
relationship, thus, we will use the ER method to create the relational schema
for this entity-relationship model. Since Reaction has a strict-one-to-many
relationship with Subscriber, we can add the primary keys of the table
Subscriber to be the attributes of the table Reaction to represent this
relationship without having another table. We name these attributes to be uname
and unum reference to the \key{username} and \key{number} in the table
Subscriber with the domain of CLOB and INT, respectively. The primary key for
Reaction is \key{id} with INT as the type. We will automatically increment the
\key{id} of the Reaction. Furthermore, Reaction has title and content as its
attributes. Since the title would be shorter than the content of a reaction, we
will use VARCHAR(100) for title and CLOB for content. The relational schema for
Reaction is
\begin{align*}
    \text{\textbf{Reaction}(} & \text{\key{id}: INT,} \\
                              & \text{title: VARCHAR(100),} \\
                              & \text{content: CLOB,} \\
                              & \text{uname: CLOB,} \\
                              & \text{unum: INT).} 
\end{align*}

Notice that ThreadR has strict-one-to-many and ISA relationships with Reaction.
As mentioned before, we use the ER method for the ISA relationship. Thus, we
will have \key{threadfrom} as the primary key of the table ThreadR to reference
the \key{id} of the current reaction. We also have an attribute threadto to
reference the \key{id} of the reaction that the current reaction is reacting to.
Since both \key{threadfrom} and threadto reference the \key{id} of the
reactions, both have the type INT and must be not null. To make sure a reaction
is not reacting to itself, we need to check that \key{threadfrom} $<>$ threadto
(\textbf{CHECK}(threadfrom $<>$ threadto)). The table ThreadR is
\begin{align*}
    \text{\textbf{ThreadR}(} & \text{\key{threadfrom}: INT,} \\
                             & \text{threadto: INT).}
\end{align*}

We also have the table for the entity ReviewR. We will have \key{reactfrom} as
the primary key for the table ReviewR, referencing the \key{id} of a reaction,
whose domain of attribute is INT. Since ReviewR also has a strict-one-to-many
relationship with Review entity, we will add the primary (and partial) keys of
the entity Review as attributes of the table ReviewR. Thus, we will have
reacteduname: CLOB, referencing the username of the subscriber whose review is
being reacted to, reactedunum: INT, referencing the number of the subscriber
whose review is being reacted to, reactedrevision: INT, referencing the revision
number of the review. We will need to add some more attributes to the table
ReviewR since Review is also a weak entity of the entity Film (in Part 2). We
will have reactedftitle: VARCHAR(100), reactedfyear: INT, reactedfcreator: INT
as attributes, referencing the \key{title}, \key{year}, \key{creator} of the
table Film in Part 2. These will be explained more in the table Review and the
table Film Part 2. The table ReviewR is
\begin{align*}
    \text{\textbf{ReviewR}(} & \text{\key{reactfrom}: INT,} \\
                             & \text{reacteduname: CLOB,} \\
                             & \text{reactedunum: INT,} \\
                             & \text{reactedrevision: INT,} \\
                             & \text{reactedftitle: VARCHAR(100),} \\
                             & \text{reactedfyear: INT,} \\
                             & \text{reactedfcreator: INT).}
\end{align*}

For the entity Review, we will have a table Review. Since Review is a weak
entity of Subscriber, we will have the primary keys of the table Subscriber as
part of the primary keys of the table Review. Hence, we will have \key{uname}:
CLOB and \key{unum}: INT for the table Review, referencing the \key{username}
and \key{number} of the table Subscriber. We also have \key{revision} as the
partial key of the table Review. Since it store the version of the review, it
has INT as the domain. We will also have attributes score of type INT and
timestamp whose domain is TIMESTAMP. Notice that Review is also a weak entity of
the table Film in Part 2. Thus, we will need to add primary keys of the table
Film as part of the primary keys of the table Review. Thus, we will have
\key{ftitle}: VARCHAR(100), \key{fyear}: INT, \key{fcreator}: INT as primary
keys of the table Review, referencing the \key{title}, the \key{year}, and the
\key{creator} of the table Film. The choice of the domain of attributes will be
explained in Part 2. Therefore, the primary key of the Review is (\key{uname},
\key{unum}, \key{ftitle}, \key{fyear}, \key{fcreator}, \pkey{revision}). All of
the keys and attributes should be not null since they are necessary to create a
review. Furthermore, we need to check that the score should be between 0 and 10
(\textbf{CHECK}(0 $<=$ score AND score $<=$ 10)). The table Review is
\begin{align*}
    \text{\textbf{Review}(} & \text{\key{uname}: CLOB,} \\
                            & \text{\key{unum}: INT,} \\
                            & \text{\key{ftitle}: VARCHAR(100),} \\
                            & \text{\key{fyear}: INT,} \\
                            & \text{\key{fcreator}: INT,} \\
                            & \text{\key{revision}: INT,} \\
                            & \text{score: INT,} \\
                            & \text{timestamp: TIMESTAMP).}
\end{align*}

Notice that VideoReview and TextReview have ISA relationships with Review, we
will use the ER method to do represent these relationships. Thus, in the table
VideoReview, we will have (\key{uname}, \key{unum}, \key{ftitle}, \key{fyear},
\key{fcreator}, \pkey{revision}) as the primary key with the explanation as the
above. In addition, we will have one more attribute, named video. Since videos
need to be store as binary files and are often large, we will use BLOB for video
attribute, this attribute should not be null as well. The table for VideoReview
is
\begin{align*}
    \text{\textbf{VideoReview}(} & \text{\key{uname}: CLOB,} \\
                                 & \text{\key{unum}: INT,} \\
                                 & \text{\key{ftitle}: VARCHAR(100),} \\
                                 & \text{\key{fyear}: INT,} \\
                                 & \text{\key{fcreator}: INT,} \\
                                 & \text{\key{revision}: INT,} \\
                                 & \text{video: BLOB).}
\end{align*}

Similarly, the table TextReview has (\key{uname}, \key{unum}, \key{ftitle},
\key{fyear}, \key{fcreator}, \pkey{revision}) as the primary key and one more
attribute description with the domain as CLOB and the attribute should not be
null.
\begin{align*}
    \text{\textbf{TextReview}(} & \text{\key{uname}: CLOB,} \\
                                & \text{\key{unum}: INT,} \\
                                & \text{\key{ftitle}: VARCHAR(100),} \\
                                & \text{\key{fyear}: INT,} \\
                                & \text{\key{fcreator}: INT,} \\
                                & \text{\key{revision}: INT,} \\
                                & \text{description: CLOB).}
\end{align*}


\textbf{Part 2: The film information}

The table Person has the primary key \key{id}. We will let the domain of
\key{id} to be INT and let it be auto-generated. The name of the person should
be stored in CLOB and we will store birthdate in DATE type. While \key{id} and
name should be not null, birthdate can be null. The relational schema for Person
is
\begin{align*}
    \text{\textbf{Person}(} & \text{\key{id}: INT,} \\
                            & \text{name: CLOB,} \\
                            & \text{birthdate: DATE).}
\end{align*}

For the table Film, we have the \key{title}, \key{year} and \key{creator} as the
primary keys. Since the title of the film is usually short, we will use
VARCHAR(100). We will use INT to store the \key{year}. Since the \key{creator}
references the \key{id} column in the table Person, we will use INT for
\key{creator}. We will use INTERVAL to specify the domain of the attribute
duration and DECIMAL for the attribute budget since the budget can have half a
dollar. We also need to check that the year should be before or equal to 2021
(\textbf{CHECK} year $<=$ 2021). Since a released film should have title, year
of release, the creator, the duration and the budget, all the keys and
attributes should not be null. The relational schema for Film is
\begin{align*}
    \text{\textbf{Film}(} & \text{\key{title}: VARCHAR(100),} \\
                          & \text{\key{year}: INT,} \\
                          & \text{\key{creator}: INT,} \\
                          & \text{duration: INTERVAL,} \\
                          & \text{budget: DECIMAL).} \\
\end{align*}

For the table Role, since a person can have multiple role and a film can have
multiple people with their roles, we cannot use them as primary keys. Thus, we
will have a primary key \key{rid} to represent the role number. It will have the
type INT and is auto-generated. The table also have pid: INT, referencing the
\key{id} of the Person table


\end{document}